\section{Problem Definition}
\label{sec:problem}
Our goal is to generate a random graph with a given motif distribution $D$,
where each graph with correct motif counts is chosen with equal probability.

This is a difficult problem, so we solve the easier problem of generating
an arbitrary graph with motif distribution $D$.  Given a solution to this
problem, we may be able to solve the original problem by finding 
transformations that preserve the motif distribution $D$, then showing
that every graph with motif distribution $D$ can be obtained through a
sequence of such transformations.  However, this is very challenging and we
relegate it to future work.

Finding an arbitrary graph with motif distribution $D$ is not always
possible and may be NP-hard.  Therefore, we try to find a graph with a
motif distribution that closely approximates $D$, where "closeness" is
defined by the average relative error between the graph's motif counts and
the desired motif counts (Equation~\ref{eqn:avgRelativeError}).

\begin{framed}
\vpara{Input:} A motif distribution $D$, where each motif has at most $4$ vertices.

\vpara{Output:} A graph $G$ with a motif distribution that closely
approximates $D$.
\end{framed}
\\\\
We first focus on an easier problem, where we are given both a motif
distribution and the real-life network it corresponds to.  This problem is
easier since it allows us to use the degree distribution of the original
network.  It also guarantees that a solution exists, which is not true for
all motif distributions.  (For example, there are no graphs with two nodes
and three triangles.)  In solving this problem we should not just return
the original graph, since that method cannot be easily extended to solve
the previous problems.

If our purpose is to compare null models to real data sets, it suffices to
solve the easier problem, since we have the data required to create the
model.  If we want to solve the harder problem, we can assume that the
degree distribution follows a power law and use the motif counts to predict
the exponent.  Then we can use the solution to the easier problem to solve
the harder problem.  We predict the exponent in Section~\ref{sec:alpha}.


%The problem formulation is different from other graph generation problems
%\cite{erdds1959random, watts1998collective, albert2002statistical,
%newman2009random, molloy1995critical}, as in this paper we focus on
%generating graphs based on the motif distribution.
%Motif distributions are frequently used for analyzing large graphs in the
%social sciences.
