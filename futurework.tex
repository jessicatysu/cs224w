\section{Future work}
\label{sec:futurework}

So far we know how to (approximately) generate graphs with given motif counts, assuming we know the degree distribution of the graph.  We can also predict properties of the degree distribution from the motif counts.  The next logical step is to use the predicted degree distribution to generate a graph.

We also hope to improve the random restarts method by increasing the number of unsuccessful rewires required for a random restart.  We will also tweak the number of random rewirings that happen at each random restart.  Currently those numbers are $|E|$ and $|E|/8$, but we intend to run simulations for several different parameter values.

%Our current algorithms assume that we know the degree distribution of the graph.  This is fine if we have the graph of a real-world social network and we are trying to build a model to compare it to.  However, it fails if we are given the motif counts only.

%Fortunately, social networks tend to have power-law degree distributions, which means their distributions can be described by a single parameter $\alpha$, where $\alpha$ is the magnitude of the exponent.  (We also need a normalization constant, which can be found from the number of edges.)  For the final paper we will build a model to predict $\alpha$ from the motif counts.  We think we can do this because graphs with a lot of high-edge motifs should be denser, so the degree distribution should have a heavier tail.

%Models to try include linear regression, neural networks, and stochastic gradient descent.
