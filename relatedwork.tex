\section{Related work}
\label{sec:related}

\vpara{Motifs.} Graph motifs are an important local property which are defined as recurrent and statistically significant subgraphs or patterns. Many researchers have studied graph motifs. 
Milo et al.~\cite{milo2002network} uses network motifs to uncover structural design principles in complex networks.
Shen-Orr et al.~\cite{shen2002network} systematically detects network motifs in one of the best-characterized regulation networks, that of direct transcriptional interactions in Escherichia coli. 
Alon et al.~\cite{alon2007network} reviews network motifs and their functions, with an emphasis on experimental studies.
Recently, Bhuiyan et al.~\cite{bhuiyan2012guise} proposes a method called \textit{GUISE}, which uses a Markov Chain Monte Carlo (MCMC) sampling method for constructing the approximate motif distribution of a large network.

\vpara{Generating random graphs.} Generating random graphs is an important problem in social network analysis. 

Most work in this area has involved degree distributions. 
Molloy et al.~\cite{molloy1995critical} provides a model for generating random graphs with a given degree sequence. 
Rao et al.~\cite{rao1996markov} proposes an MCMC based model using switches along alternating cycles for generating random graphs.
Bayati et al.~\cite{bayati2010sequential} presents a nearly-linear time algorithm for counting and randomly generating simple graphs with a given degree sequence in a certain range. 

Our approach is inspired by Milo et al.~\cite{milo04random}, who proposes a method of generating a random graph with a prescribed degree sequence.  He begins by generating an arbitrary graph with that degree sequence.  (This can be done by giving each node a certain number of ``half-edges,'' then connecting them randomly to form the edges of the graph.)  Once this graph is generated, he chooses pairs of edges at random and swaps the endpoints, repeating this step until the graph is sufficiently ``randomized.''

\begin{algorithm}
\caption{Milo's approach for generating random graphs with prescribed degree sequences.}
\label{algorithm:milo}
\begin{algorithmic}
Generate graph $G = (V, E)$ with required degree sequence\\
\While{\text{Markov chain displays insufficient mixing}}{
    Choose $e_1, e_2 \in E$ at random\\
    Add edges $(e_1.Src, e_2.Dst)$, $(e_2.Src, e_1.Dst)$\\
    Delete edges $(e_1.Src, e_1.Dst)$, $(e_2.Src, e_2.Dst)$\\
}
\end{algorithmic}
\end{algorithm}

This ``rewiring'' step is innovative because it preserves both the number of edges and the degree of each node.  For this reason we use the same rewiring step in our algorithm.